%%%%%%%%%%%%%%%%%%%%%%%%%%%%%%%%%%%%%%%%%
% Twenty Seconds Resume/CV
% LaTeX Template
% Version 1.1 (8/1/17)
%
% This template has been downloaded from:
% http://www.LaTeXTemplates.com
%
% Original author:
% Carmine Spagnuolo (cspagnuolo@unisa.it) with major modifications by 
% Vel (vel@LaTeXTemplates.com)
%
% License:
% The MIT License (see included LICENSE file)
%
%%%%%%%%%%%%%%%%%%%%%%%%%%%%%%%%%%%%%%%%%

%----------------------------------------------------------------------------------------
%	PACKAGES AND OTHER DOCUMENT CONFIGURATIONS
%----------------------------------------------------------------------------------------

\documentclass[letterpaper]{twentysecondcv} % a4paper for A4
\usepackage{enumitem}
\usepackage{xcolor}
\usepackage[colorlinks]{hyperref}
%----------------------------------------------------------------------------------------
%	 PERSONAL INFORMATION
%----------------------------------------------------------------------------------------

% If you don't need one or more of the below, just remove the content leaving the command, e.g. \cvnumberphone{}



\cvname{Pratik Bhangale} % Your name
%\cvjobtitle{Gadepalli} % Job title/career


\cvmail{bhangalepratik8@gmail.com}
\cvphone{+91-8960742030}
\cvlocation{https://www.google.com/maps/place/19.845169,75.239629}
\cvgithub{https://github.com/pratikab}
\cvlinkedin{https://www.linkedin.com/in/pratikab/}
 % Short address/location, use \newline if more than 1 line is required

%----------------------------------------------------------------------------------------

\begin{document}

%----------------------------------------------------------------------------------------
%	 Education
%----------------------------------------------------------------------------------------

\Education{\textbf{B.tech in Computer Science} \\ Minor in Microelectronics\\ IIT Kanpur | 2018 | GPA : 7.6/10
\newline \newline \textbf{Class XII in Maharashtra Board}\newline  Tanwani Jr College | 2014 | 87.54 \%
\newline \newline \textbf{Class X in Maharashtra Board}\newline  Tanwani Eng. Sch. | 2012 | 98.00 \%} % To have no Education section, just remove all the text and leave \Education{}

%----------------------------------------------------------------------------------------
%	 SKILLS
%----------------------------------------------------------------------------------------
% Skill bar section, each skill must have a value between 0 an 6 (float)
\skills{\textbf{Programming}:\\ \textbullet{} C/C++ \textbullet{} Python \textbullet{} Matlab \textbullet{} Verilog \\ \textbullet{} Shell Script \textbullet{} SQL
\newline  \textbf{Libraries and Tools:} \\\textbullet{} Tensorflow \textbullet{} Keras \textbullet{} PyTorch\\ \textbullet{} OpenCV \textbullet{} NLTK \textbullet{} Django \textbullet{} GIT\\ \textbullet{} Perforce \textbullet{} Makefile \textbullet{} LATEX \textbullet{} GDB\\ \textbullet{} Altium PCB designer \\
\textbf{Operating Systems::}\\
\textbullet{}Linux(Arch/Debian) \textbullet{}Windows
}

\Coursework{
\textbullet{} Reinforcement Learning (MOOC)\\
\textbullet{} Natural Language Processing (A)\\
\textbullet{} Introduction to  Machine learning (A)\\
\textbullet{} Computer Architecture (A)\\
\textbullet{} Computer Systems and Security (A)\\
\textbullet{} Computer Networks\\
\textbullet{} Compiler Techniques \\
\textbullet{} Operating Systems \\
\textbullet{} Database Management Systems \\
\textbullet{} Data Structures \& Algorithms \\
\textbullet{} Probability \& Statistics \\
\textbullet{} Digital Electronics\\
\textbullet{} Introduction to Microelectronics\\
}


\PoR{
\textbullet{} Coordinator, Electronics Club IITK\\
\hspace{1.5cm}\location{Mar '16 - Feb '17}\\
\textbullet{} Sub-head, Team Robocon IITK\\
\hspace{1.5cm}\location{Jul '15 - Mar '16}
}
\Interests{
\textbullet{} Machine Learning and AI\\
\textbullet{} Computer Systems\\
\textbullet{} Microelectronics
}

%------------------------------------------------

% Skill text section, each skill must have a value between 0 an 6


%----------------------------------------------------------------------------------------
\makeprofile % Print the sidebar

%----------------------------------------------------------------------------------------
%	 INTERESTS
%----------------------------------------------------------------------------------------


%----------------------------------------------------------------------------------------
%	 EDUCATION
%----------------------------------------------------------------------------------------
\setlist[itemize]{noitemsep, topsep=0pt,leftmargin=*}
\section{PROFESSIONAL EXPERIENCE}
\subsection{SOFTWARE ENGINEER}\subsubsection{\href{https://research.samsung.com/sri-n}{Samsung Research Institute, Noida}}  \hfill{} June '18-Present
\begin{itemize}
    \item Part of Android kernel development \& Embedded devices bring up team for Qualcomm and Exynos Chipsets. Currently working on Android Pie \& Q projects
    \item Optimize and port boot loader and device kernel for latest Samsung mobile devices
    \item Designed \& Implemented an automated bringup process to speed up project delivery
    \item Samsung Software Competency(SWC) certified Professional coder
\end{itemize}
\subsection{SOFTWARE INTERN}\subsubsection{\href{https://research.samsung.com/sri-n}{Samsung Research Institute, Noida}}\hfill{}  May '17 - July '17
\begin{itemize}
    \item Developed two factor authentication system for Fingerprint called BioHashing
\item Detected core-points in a Fingerprint using Gradient Orientation Map \& hashed it to multi-dimensional vector to generate a BioHash for preventing misuse of biometrics
% \item Tested the system on FVC2002 dataset and achieved 4.34\% False Rejection Rate (FRR) at 0.0017\% False Acceptance Rate (FAR) and proved that it can be deployed.
\end{itemize}

\section{ACHIEVEMENTS}
%\vspace{2mm}
\begin{itemize}
% 	\item Cleared JEE Mains and Advanced 2014 with percentile 99.98
% 	\item Recipient of Kishor Vaigyanik Protsahan Yojana (\href{http://www.kvpy.iisc.ernet.in/main/index.htm}{KVPY}) scholarship since 2014 by Indian Institute Science, Banglore
	\item Recipient of Kishor Vaigyanik Protsahan Yojana (\href{http://www.kvpy.iisc.ernet.in/main/index.htm}{KVPY}) scholarship by IISc, Banglore
	\item Appeared in Indian National Chemistry Olympiad (\href{http://chem.hbcse.tifr.res.in/in-india/}{INChO}) among top 300 students
	\item Recipient of National Talent Search Examination (\href{http://www.ncert.nic.in/programmes/talent_exam/index_talent.html}{NTSE}) scholarship by NCERT
	\item Second Runners Up in \href{http://www.roboconindia.com/}{ABU Robocon India} 2016 representing IIT Kanpur %among 64 participating universities.
	\item First place in Electromania, \href{https://techkriti.org/}{Techkriti} (Annual Technical Festival of IIT Kanpur)
\end{itemize}
\section{RESEARCH AND PROJECTS}
\subsection{deCAPTCHA}\subsubsection{Prof. Purushottam Kar, IITK | \faGithub{ \href{https://github.com/pratikab/deCAPTCHA}{code}}} \hfill{} July '17 - Nov '17

\begin{itemize}
\item The objective was to build efficient algorithms to break squirrel mail captchas
\item Used K-means clustering and Image Processing tools to overcome heavy noises and clutters cutting through captcha text that makes it difficult for CNN models to work
\item Implemented CNN models in PyTorch for character recognition which proved to be 98\% accurate for character recognition and 85\% for entire captcha
\end{itemize}

\subsection{MACHINE COMPREHENSION}
\subsubsection{Prof. Harish Karnick, IITK | \faGithub{ \href{https://github.com/2ashish/NLP-Answering-Reading-Comprehension}{code}}} \hfill{} Dec '17 - Apr '18

\begin{itemize}
\item Objective was to help machine understand the comprehension and answer questions
\item Word and character embeddings of comprehensions and questions were passed into Bi-Directional LSTM layers to extract information vector from tokens of the text
\item Question Attention Layer was generated to find out important information of question, which later is multiplied with comprehension to get probable positions of answer
\item Model proved to be 97\% effective on bAbI dataset by Facebook
\end{itemize}

\subsection{COMPUTER ARCHITECTURE}
\subsubsection{Prof. Mainak Chaudhuri, IITK | \faGithub{ \href{https://github.com/Naruto8/Project422}{code}}} \hfill{} Dec '16 - Apr '17

\begin{itemize}
\item Designed and simulated state-of-the-art cache replacement policies like Least Recently Used(LRU), Static Re-reference Interval Prediction(SRRIP) and Signature based Hit Predictor (SHiP) using Pintool by Intel on 16 set-associative L3 Cache
\item Implemeted Pipelined MIPS processor with Register Forwarding to avoid data hazards on KSIM simulator
\end{itemize}

\subsection{C TO x86 COMPILER}
\subsubsection{Prof. Amey Karkare, IITK | \faGithub{ \href{https://github.com/pratikab/Compiler-C-to-X86}{code}}} \hfill{} Dec '16 - Apr '17

\begin{itemize}
\item Developed a full fledged Compiler for a subset of C for x86 architecture in python
\item Implemented Lexical Analyzer, Parser, Three address code \& Assembly code generator
\item The compiler supported basic arithmetic, conditionals, mutual recursion, parameterized functions, global declarations and scope handling
\end{itemize}


\subsection{ABU Robocon '16}
\subsubsection{Prof. Bhaskar Dasgupta | IITK} \hfill{} Sept '15 - Mar '16
\begin{itemize}
\item Designed \& fabricated one autonomous and one semi-autonomous robots which could complete tasks like pole climbing, object placing, line and wall following to participate in ABU Robocon 2016 competition
\end{itemize}

% \runsubsection{FPGA based Audio Mixer}\\
% \location{\textbf{\href{http://students.iitk.ac.in/eclub}{Electronics Club}} | IITK \hfill{} May '15 - June '16}
% \begin{itemize}
% \item Developed an Audio Mixer device using FPGA taking Audio signal as input and producing different sound effects like Echo, Pitch change, etc.
% \item Worked on parallel processing, serial communication (SPI) on Spartan 6 MOJO FPGA, using Verilog as Hardware Description Language
% \end{itemize}

\subsection{MINI PROJECTS}
\begin{itemize}
\item Flappy Bird AI using Q-Learning and Reinforcement Learning using Keras %| \subsubsection{\faGithub{ \href{https://github.com/pratikab/flappy_ai}{code}}}
\item Hotel automation system using Django and SQlite
\item File Sharing with user authentication system on network using Socket programming
\item OS process sceduling, system calls, page allocation implementation for NachOS
\item Chat Bot with slack and hubot integration to do personalized tasks like setting reminders, fetching emails, adding notes and setting up team meetings
\end{itemize}


%----------------------------------------------------------------------------------------
%	 OTHER INFORMATION
%----------------------------------------------------------------------------------------




%----------------------------------------------------------------------------------------
%	 SECOND PAGE EXAMPLE
%----------------------------------------------------------------------------------------

%\newpage % Start a new page

%\makeprofile % Print the sidebar

%\section{Other information}

%\subsection{Review}

%Alice approaches Wonderland as an anthropologist, but maintains a strong sense of noblesse oblige that comes with her class status. She has confidence in her social position, education, and the Victorian virtue of good manners. Alice has a feeling of entitlement, particularly when comparing herself to Mabel, whom she declares has a ``poky little house," and no toys. Additionally, she flaunts her limited information base with anyone who will listen and becomes increasingly obsessed with the importance of good manners as she deals with the rude creatures of Wonderland. Alice maintains a superior attitude and behaves with solicitous indulgence toward those she believes are less privileged.

%\section{Other information}

%\subsection{Review}

%Alice approaches Wonderland as an anthropologist, but maintains a strong sense of noblesse oblige that comes with her class status. She has confidence in her social position, education, and the Victorian virtue of good manners. Alice has a feeling of entitlement, particularly when comparing herself to Mabel, whom she declares has a ``poky little house," and no toys. Additionally, she flaunts her limited information base with anyone who will listen and becomes increasingly obsessed with the importance of good manners as she deals with the rude creatures of Wonderland. Alice maintains a superior attitude and behaves with solicitous indulgence toward those she believes are less privileged.

%----------------------------------------------------------------------------------------
\updateinfo{}
\end{document} 
